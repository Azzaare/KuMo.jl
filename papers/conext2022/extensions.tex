%% summary and future work

In this paper, we have presented the cloud morphing vision for
future distributed cloud services,
proposed a cost function based resource allocation model,
and presented the basic idea for idle-resource pooling.
We are planning to refine the proposed model and developing a working
prototype system.

For model refinement,
the current model is simplistic. We will add bidirectional links,
multiple users per job, interactions among jobs, and so on.
We did not consider dependency among microservice jobs but it would be
necessary to investigate the impact of the interplay of microservice
jobs ~\cite{Suresh-SOA-SOCC2017}.

For prototype development,
we need realistic microservice workload models.
Also, it is necessary to develop mechanisms and protocols for
discovering available resources~\cite{Albrecht2008} and exchanging
cost information.
A path selection mechanism is also needed when assigning a job,
probably using SRv6 or other source routing mechanism.

There exists a rich area for further research:
{\bf How to tune the cost function:}  It may not be straightforward
to obtain expected results by tuning the cost function, as it may
induce side effects.
It would be interesting to apply reinforcement learning to cost
tuning.

{\bf Hierarchical Configurations:} It would be better to use
hierarchical configurations for scalability and for management
purposes.
For example, a distributed datacenter model can consist of
the inter-DC layer with DC-level resources and the intra-DC layer
with rack-level resources.

{\bf Inter-cloud:}
The proposed system was originally designed mainly for a single
administrative domain.
The hierarchical model, however, naturally extends to an inter-cloud
model in which different cloud systems are federated.
The mechanism of the idle resource pooling allows utilizing external
clouds when only needed.

{\bf Crowdsourcing resource supply:}
It would be possible to crowdsource the role of resource providers. 
Then, we need a monetary charging, authentication and authorization
mechanisms to add externally provided resources to the resource pool.

{\bf Data Migration:}
The location of data is critical for the pseudo cost so that data
placement would play an important role for cloud morphing. 
A possible approach is to migrate data at coarser timescale, based
on usage history.
Another possibility is design a new distributed storage system
suitable for the cloud morphing model in which (cached) data can be
easily moved.

{\bf L2 Paths:}
idle-resource pooling is not used for network links in this paper but
it is possible to use it for dynamically creating Layer 2 paths (e.g.,
lightpath switching over WDM networks).

Our simulation tool is publicly available from (URL removed for
review).
