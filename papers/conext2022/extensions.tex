
The basic resource management model has been presented.
But there exists a rich area for further research, as well as
necessary solutions for building a working prototype system.

\begin{description}

\item[Model Refinement:]  The current model is simplistic. We will add
  bidirectional links, multiple users per job, interactions among
  jobs, and so on.

\item[Hierarchical Configurations:] It would be better to use
  hierarchical configurations for scalability and for management
  purposes.
  For example, a distributed datacenter model can consist of
  the inter-DC layer with DC-level resources and the intra-DC layer
  with rack-level resources.

\item[Inter-cloud:]
  The hierarchical model naturally extends to an inter-cloud model in which
  different cloud systems are federated.
  The mechanism of the idle resource pooling allows utilizing external
  clouds when only needed.

\item[Data Migration:]
  The location of data is critical.
  A possible approach is to migrate data at coarser timescale, based
  on usage history.
  Another possibility is design a new distributed storage system in
  which (cached) data can be easily moved.

\item[L2 Paths:] It is possible to use idle resource pooling for dynamically
  creating Layer 2 paths (e.g., light path switching over WDM networks).

\item[Resource discovery and cost exchange protocols:]
  It is necessary to develop mechanisms and protocols for finding
  available resources and their costs~\cite{Albrecht2008}..

\item[Source Routing:] We assumed that it is possible to select network paths
  when assigning a job, probabaly using SRv6 or other source routing mechanism.

\end{description}
