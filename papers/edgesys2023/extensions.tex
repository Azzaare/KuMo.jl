%% summary and future work

We have presented the cloud morphing vision for future distributed
cloud services,
proposed a resource allocation model based on cost functions,
and presented how the idle-resource pooling works.
We are planning to refine the proposed model and develop a working
prototype system.

For the model refinement,
the current model is simplistic so that we will add bidirectional
communication costs,
multiple users per job, interactions among jobs, and other features.
We did not consider dependency among microservice jobs, but it would be
necessary to investigate the impact of the interplay of microservice
jobs ~\cite{Suresh-SOA-SOCC2017}.

For the prototype development,
we need realistic future microservice workload models.
Also, it is necessary to develop mechanisms and protocols for
discovering available resources~\cite{Albrecht2008} and exchanging
cost information.
A path selection mechanism is also needed when assigning a job,
probably using a source routing mechanism.

There exists a rich area for further research:
%one topic is {\bf auto-tuning of the cost functions}.
%It may not be straightforward to obtain expected results by tuning the
%convex cost function due to the interactions between load-balancing
%and idle-resource pooling.
%It would be interesting to apply reinforcement learning to cost
%tuning.
one topic is {\bf hierarchical configurations}.
A hierarchical system model would be preferred for scalability
and for management purposes.
For example, a distributed datacenter model can consist of
the inter-DC layer with DC-level resources and the intra-DC layer
with rack-level resources.
%The proposed system was originally designed mainly for a single
%administrative domain.
The hierarchical model naturally extends to
an {\bf inter-cloud model} in which different cloud systems are
federated.
The mechanism of the idle-resource pooling allows utilizing external
clouds only when needed.

Further, it would be possible to {\bf crowdsource the resource supply}
at the edge. 
Then, we need a monetary charging, authentication and authorization
mechanisms to add externally provided resources to the resource pool.

The location of data is critical for the pseudo cost so that
{\bf data placement} would play an important role for cloud morphing. 
A possible approach is to migrate data at a coarser timescale, based
on usage history.
Another possibility is to design a new distributed storage system
suitable for the cloud morphing model in which (cached) data can be
easily moved.

The idle-resource pooling is not used for network links in this paper, but
it is possible to use it for dynamically making {\bf Layer 2 paths}
(e.g., lightpath switching over WDM networks).

We believe that future distributed heterogeneous clouds need a new
paradigm for resource management, and hope this work will stimulate
other research in the field.
